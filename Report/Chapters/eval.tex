The description of the analysis where were pefromed during the
research and helped to achieve the objective of the research.

\section{Stastical Hypothesis Testing}
The following independent unpaired stastical testing was peformed using 
IBM SPSS software. The ${H_o}$(null hypothesis) = there is no difference in 
predicting the class of pigmented skin lesions by automated system and 
medical professional. On the otherhand ${H_1}$(alternative hypothesis) = there is 
a difference in time required to predict pigmented skin lesions.
\begin{figure}[!htp]
    \includegraphics[width=15cm]{Images/ttest.png}
    \caption{Independent sample t-test}
    \label{fig:istt}
\end{figure}

The figure \ref{fig:istt} shows the results obtained by comparing the 
means of the time required by the automated system and medical professional to predict pigmented skin 
lesions. The group 1 in the test is reffered to the automated system and 
group 2 is reffered to medical professionals. Based on the p value obatined from the 
test which is 0.000 as in \ref{fig:istt} in the Sig(2-tailed) the null 
hypothesis is failed. Thus, it can be concluded that there is a significant 
difference in time required to peform diagnosis. The purposed automated machine 
is more time efficient to predict.

\section{Confusion Matrix}
\begin{figure}[!htp]
    \includegraphics[width=\textwidth]{Images/cm.png}
\end{figure}
Confusion Matrix also known as the error matrix is the table to describe the 
perofarmance of automated system given true values of classification classes \citep{geekforgeeks}. The 
confusion matrix also allows the misclassified predicted labels by the automated 
and helps to evaluate the performance measures \citep{geekforgeeks}.

\subsection{Recall and Precision of Automated System}
\begin{center}
    Recall = ${ \frac{True Positive} {False Negative + True Positive}}$
    \label{e1}
\end{center}

\begin{center}
    Precision = ${ \frac{True Positive} {True Positive + False Positive}}$
\end{center}

The equation above are be used to obtain the precision and recall values of the 
convolutional neural network. The Precision and recall values are further used 
to compute the f-measures also known as the f1score to evaluate the performance of 
automated system. \\ 
\begin{center}
    F-measure = ${\frac{2 * Recall * Precision}{Recall + Precision}}$
\end{center}

\subsection{Confusion Matrix and F-measure of the Neural network}
\begin{figure}[!htp]
    \includegraphics[width=\textwidth]{Images/cma.png}
    \caption{Error Matrix}
    \label{fig:ErrorMatrix}
\end{figure}

The error matrix \ref{fig:ErrorMatrix} was plotted using heatmap in matplotlib
visualisation library.