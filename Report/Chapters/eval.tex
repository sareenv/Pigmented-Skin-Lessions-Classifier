The analysis was performed based on the data collected from the participants from the medical background. 
The evaluation section focuses on test the time efficiency of the artificial neural network in predicting pigmented skin lesions. Furthermore, the confusion matrix and f1 scores are used to analyse the performance of the classification.

\section{Stastical Hypothesis Testing}
The following independent unpaired stastical testing was peformed using 
IBM SPSS software. The ${H_o}$(null hypothesis) = there is no difference in 
predicting the class of pigmented skin lesions by automated system and 
medical professional. On the otherhand ${H_1}$(alternative hypothesis) = there is 
a difference in time required to predict pigmented skin lesions.
\begin{figure}[!htp]
    \includegraphics[width=14cm]{Images/ttest.png}
    \caption{Independent sample t-test}
    \label{fig:istt}
\end{figure}
\pagebreak

The figure \ref{fig:istt} shows the results obtained by comparing the 
means of the time required by the automated system and medical professional to predict pigmented skin 
lesions. The group 1 in the test is reffered to the automated system and 
group 2 is reffered to medical professionals. Based on the p value obatined from the 
test which is 0.000 as in \ref{fig:istt} in the Sig(2-tailed) the null 
hypothesis is failed. Thus, it can be concluded that there is a significant 
difference in time required to peform diagnosis. The purposed automated machine 
is more time efficient to predict.

\section{Confusion Matrix}
\begin{figure}[!htp]
    \includegraphics[width=\textwidth]{Images/cm.png}
\end{figure}
Confusion Matrix also known as the error matrix is the table to describe the 
perofarmance of automated system given true values of classification classes \citep{geekforgeeks}. The 
confusion matrix also allows the misclassified predicted labels by the automated 
and helps to evaluate the performance measures \citep{geekforgeeks}.

\subsection*{Recall and Precision of Automated System}
\begin{center}
    Recall = ${ \frac{True Positive} {False Negative + True Positive}}$
\end{center}

\begin{center}
    Precision = ${ \frac{True Positive} {True Positive + False Positive}}$
\end{center}

The equations above are be used to obtain the precision and recall values of the 
convolutional neural network. The Precision and recall values are further used 
to compute the f-measures also known as the f1score to evaluate the performance of 
automated system. \\ 
\begin{center}
    F-measure = ${\frac{2 * Recall * Precision}{Recall + Precision}}$
\end{center}

\subsection*{Confusion Matrix and F-measure of the Neural network}
\begin{figure}[!htp]
    \includegraphics[width=\textwidth]{Images/cma.png}
    \caption{Error Matrix}
    \label{fig:ErrorMatrix}
\end{figure}

The confusion matrix was computed using the sklearn learn library which requires the predicted values and 
actual true labels for each pigmented skin lesions. Futhermore the confusion matrix  was plotted using heatmaps
in the matplotlib visualisation library.
\pagebreak

\begin{figure}[!htp]
    \includegraphics[width=\textwidth]{Images/f11.png}
    \caption{Classification Report of automated system}
    \label{fig:f11}
\end{figure}

The figure above shows the analysis of the automated system in an attempt to predict pigmented skin lesions. The report above shows precision, recall and f1 score values which indicates the performance in predicting each category of pigmented skin lesions. All further analysis to compare the model performance with participants are evaluated based on the f1 scores for each category of pigmented skin lesion.
\pagebreak

\section{Most common method for diagnosing skin lesions}
\begin{figure}[!htp]
    \includegraphics[width=15cm]{Images/dia.png}
    \caption{Most Common Diagnosis}
    \label{fig:commonPigmentedSkinLesions}
\end{figure}

Figure \ref{fig:commonPigmentedSkinLesions} shows the results obtained from the 
participants on most common method of diagnosing skin lesions. The half majority of the 
overall participants confirmend that ABCD rule is used to detect pigmented skin lesions. 
Whereas 40\% and 10\% population believed in diagnosing using 7 checklist principal 
method and combination of both the technique respectively.  
\section{Automated System can help to patients to diagnose book prior appointments}
\begin{figure}[!htp]
    \includegraphics[width=\textwidth]{Images/help.png}
    \label{fig:help}
\end{figure}

Based on the results obtained from participants as shown in the bar chart 
above majority of the participants mostly agree that automated system can 
help patients to book prior appointments and to predict classes of pigmented skin lesions.

\pagebreak
\section{Classification comparison with Participants}

\subsection*{Classification comparison with Participant 1 }

\begin{figure}[!htp]
    \includegraphics[width=\textwidth]{Images/p1.png}
    \caption{Confusion Matrix from Participant 1}
    \label{fig:f11}
\end{figure}

\begin{figure}[!htp]
    \includegraphics[width=\textwidth]{Images/p1r.png}
    \caption{Classification Report of participant 1}
    \label{fig:f11}
\end{figure}

\pagebreak

\begin{figure}[!htp]
    \includegraphics[width=\textwidth]{Images/a1.png}
    \caption{Confusion Matrix from Automated System prediction}
    \label{fig:f11}
\end{figure}

\begin{figure}[!htp]
    \includegraphics[width=\textwidth]{Images/a1r.png}
    \caption{Classification Report of Automated System}
    \label{fig:f11}
\end{figure}



\pagebreak
\subsection*{Classification comparison with Participant 2 }

\begin{figure}[!htp]
    \includegraphics[width=\textwidth]{Images/p2.png}
    \caption{Confusion Matrix from Participant 2}
    \label{fig:f11}
\end{figure}

\begin{figure}[!htp]
    \includegraphics[width=\textwidth]{Images/p2r.png}
    \caption{Classification Report of participant 2}
    \label{fig:f11}
\end{figure}

\pagebreak

\begin{figure}[!htp]
    \includegraphics[width=\textwidth]{Images/a2.png}
    \caption{Confusion Matrix from Automated System prediction}
    \label{fig:f11}
\end{figure}

\begin{figure}[!htp]
    \includegraphics[width=\textwidth]{Images/a2r.png}
    \caption{Classification Report of Automated System}
    \label{fig:f11}
\end{figure}


\pagebreak
\subsection*{Classification comparison with Participant 3 }

\begin{figure}[!htp]
    \includegraphics[width=\textwidth]{Images/p3.png}
    \caption{Confusion Matrix from Participant 3}
    \label{fig:f11}
\end{figure}

\begin{figure}[!htp]
    \includegraphics[width=\textwidth]{Images/p3r.png}
    \caption{Classification Report of participant 3}
    \label{fig:f11}
\end{figure}

\pagebreak

\begin{figure}[!htp]
    \includegraphics[width=\textwidth]{Images/a3.png}
    \caption{Confusion Matrix from Automated System prediction}
    \label{fig:f11}
\end{figure}

\begin{figure}[!htp]
    \includegraphics[width=\textwidth]{Images/a3r.png}
    \caption{Classification Report of Automated System}
    \label{fig:f11}
\end{figure}

\pagebreak
\subsection*{Classification comparison with Participant 4 }

\begin{figure}[!htp]
    \includegraphics[width=\textwidth]{Images/p4.png}
    \caption{Confusion Matrix from Participant 4}
    \label{fig:f11}
\end{figure}

\begin{figure}[!htp]
    \includegraphics[width=\textwidth]{Images/p4r.png}
    \caption{Classification Report of participant 4}
    \label{fig:f11}
\end{figure}

\pagebreak

\begin{figure}[!htp]
    \includegraphics[width=\textwidth]{Images/a4.png}
    \caption{Confusion Matrix from Automated System prediction}
    \label{fig:f11}
\end{figure}

\begin{figure}[!htp]
    \includegraphics[width=\textwidth]{Images/a4r.png}
    \caption{Classification Report of Automated System}
    \label{fig:f11}
\end{figure}

\pagebreak
\subsection*{Classification comparison with Participant 5}

\begin{figure}[!htp]
    \includegraphics[width=\textwidth]{Images/p5.png}
    \caption{Confusion Matrix from Participant 5}
    \label{fig:f11}
\end{figure}

\begin{figure}[!htp]
    \includegraphics[width=\textwidth]{Images/p5r.png}
    \caption{Classification Report of participant 5}
    \label{fig:f11}
\end{figure}

\pagebreak

\begin{figure}[!htp]
    \includegraphics[width=\textwidth]{Images/a5.png}
    \caption{Confusion Matrix from Automated System prediction}
    \label{fig:f11}
\end{figure}

\begin{figure}[!htp]
    \includegraphics[width=\textwidth]{Images/a5r.png}
    \caption{Classification Report of Automated System}
    \label{fig:f11}
\end{figure}

\pagebreak
\subsection*{Classification comparison with Participant 6}

\begin{figure}[!htp]
    \includegraphics[width=\textwidth]{Images/p6.png}
    \caption{Confusion Matrix from Participant 6}
    \label{fig:f11}
\end{figure}

\begin{figure}[!htp]
    \includegraphics[width=\textwidth]{Images/p6r.png}
    \caption{Classification Report of participant 6}
    \label{fig:f11}
\end{figure}

\pagebreak

\begin{figure}[!htp]
    \includegraphics[width=\textwidth]{Images/a6.png}
    \caption{Confusion Matrix from Automated System prediction}
    \label{fig:f11}
\end{figure}

\begin{figure}[!htp]
    \includegraphics[width=\textwidth]{Images/a6r.png}
    \caption{Classification Report of Automated System}
    \label{fig:f11}
\end{figure}


\pagebreak
\subsection*{Classification comparison with Participant 7}

\begin{figure}[!htp]
    \includegraphics[width=\textwidth]{Images/p7.png}
    \caption{Confusion Matrix from Participant 7}
    \label{fig:f11}
\end{figure}

\begin{figure}[!htp]
    \includegraphics[width=\textwidth]{Images/p7r.png}
    \caption{Classification Report of participant 7}
    \label{fig:f11}
\end{figure}

\pagebreak

\begin{figure}[!htp]
    \includegraphics[width=\textwidth]{Images/a7.png}
    \caption{Confusion Matrix from Automated System prediction}
    \label{fig:f11}
\end{figure}

\begin{figure}[!htp]
    \includegraphics[width=\textwidth]{Images/a7r.png}
    \caption{Classification Report of Automated System}
    \label{fig:f11}
\end{figure}


\pagebreak
\subsection*{Classification comparison with Participant 8}

\begin{figure}[!htp]
    \includegraphics[width=\textwidth]{Images/p8.png}
    \caption{Confusion Matrix from Participant 8}
    \label{fig:f11}
\end{figure}

\begin{figure}[!htp]
    \includegraphics[width=\textwidth]{Images/p8r.png}
    \caption{Classification Report of participant 8}
    \label{fig:f11}
\end{figure}

\pagebreak

\begin{figure}[!htp]
    \includegraphics[width=\textwidth]{Images/a8.png}
    \caption{Confusion Matrix from Automated System prediction}
    \label{fig:f11}
\end{figure}

\begin{figure}[!htp]
    \includegraphics[width=\textwidth]{Images/a8r.png}
    \caption{Classification Report of Automated System}
    \label{fig:f11}
\end{figure}

\pagebreak
\subsection*{Classification comparison with Participant 9}

\begin{figure}[!htp]
    \includegraphics[width=\textwidth]{Images/p9.png}
    \caption{Confusion Matrix from Participant 9}
    \label{fig:f11}
\end{figure}

\begin{figure}[!htp]
    \includegraphics[width=\textwidth]{Images/p9r.png}
    \caption{Classification Report of participant 9}
    \label{fig:f11}
\end{figure}

\pagebreak

\begin{figure}[!htp]
    \includegraphics[width=\textwidth]{Images/a9.png}
    \caption{Confusion Matrix from Automated System prediction}
    \label{fig:f11}
\end{figure}

\begin{figure}[!htp]
    \includegraphics[width=\textwidth]{Images/a9r.png}
    \caption{Classification Report of Automated System}
    \label{fig:f11}
\end{figure}

\pagebreak
\subsection*{Classification comparison with Participant 10}

\begin{figure}[!htp]
    \includegraphics[width=\textwidth]{Images/p10.png}
    \caption{Confusion Matrix from Participant 10}
    \label{fig:f11}
\end{figure}

\begin{figure}[!htp]
    \includegraphics[width=\textwidth]{Images/p10r.png}
    \caption{Classification Report of participant 10}
    \label{fig:f11}
\end{figure}

\pagebreak

\begin{figure}[!htp]
    \includegraphics[width=\textwidth]{Images/a10.png}
    \caption{Confusion Matrix from Automated System prediction}
    \label{fig:f11}
\end{figure}

\begin{figure}[!htp]
    \includegraphics[width=\textwidth]{Images/a10r.png}
    \caption{Classification Report of Automated System}
    \label{fig:f11}
\end{figure}