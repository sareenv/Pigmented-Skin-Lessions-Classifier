\section{Required Installation and Configuration}
The research purposes a solution based on deep convolutional neural networks
With various experiments mentioned in further sections.  The experiments
require specific configuration for cuda libraries to run programs on available GPU(Graphical Processing Unit) for faster processing
and further, instructions to setup environment can be found at Tensorflow gpu installation guide \footnote[1]{\url{https://www.tensorflow.org/install/gpu}}.
Alternatively, the machine learning models can also run without installing cuda libraries on CPU but the processing will be slow.
In addition, the convolutional neural network was implemented using Python3 and Jupyter notebooks were used in these experiments which provides an appropriate interface to experiment and write markdowns.  
Jupyter notebooks can be installed by installing Anaconda \footnote[2]{https://www.anaconda.com/distribution/}.
Furthermore, data science and machine learning libraries are required such as Numpy, Keras, Matplotlib and OpenCV. 
Numpy library is required for performing mathematical operations on multi-dimensional NumPy arrays. The matplotlib library provides an interface to visualise the output results. 
The OpenCV library was used for image processing in the process of developing an automated system for classifying
pigmented skin lesions. Moreover, keras library was used to develop deep convolutional models. 
HAM,1000 (Human Against Machine, 1000) dataset was used to develop image classifier \citep{DVN/DBW86T_2018}.
The dataset primarily contains two folders which includes overall 10,000 dermatoscopic images of pigmented skin lesion. In addition, it also contains a csv file which includes the meta-information regarding pigmented skin lesions.
The required files can be downloaded from harvard dataverse webiste \footnote[3]{\url{https://dataverse.harvard.edu/dataset.xhtml?persistentId=doi:10.7910/DVN/DBW86T}}.

\pagebreak
\section{Conceptual Modelling}
The Conceptual models of the system was designed using workflow 
and UML(unified modelling language) diagrams to visualise
behavioural requirments of the automated system. 
The uml diagram was produced using visual paradigm software which provides 
interface to produce digital architecture and system design of software solution. \footnote{\url{https://www.visual-paradigm.com/}}

\subsection{Sequance UML Diagram}
\begin{figure}[!htp]
    \centering
    \includegraphics[width=\textwidth]{Images/code.png}
    \caption{Sequance Diagram}
    \label{figure:Sequance}
\end{figure}
The figure \ref{figure:Sequance} represents the sequance uml (unified modelling language) diagram of the 
system. The inital actor in the system is the user or the patient of the 
pigmented skin lesion. The diagram shows the flow of the messages among 
different lifelines in the system. The messages between lifelines outlines the intraction of
the patient with the system in sequantial fashion. Therefore, the diagram helps to capture the 
behavioural requirments of the automated system.

\section{Data Processing and Normalisation}
The information was read using pandas into the data frame, 
which is a data structure that allows storing tabular data from CSV files. 
Datasets contain unclear and hairy images of pigmented skin lesions which were manually 
removed from the dataset to enhance the quality of available data.
The CSV file contained irrelevant information such as gender and age of 
patients and unbalanced data classes data columns were dropped from the dataset. Furthermore, the dataset was divided into training and 
testing sets using \url{sklearn.model_selection.train_test_split} class in the portion of 80 per cent for 
the training dataset and 20 per cent of testing datasets. The next step towards to preparing the dataset was reading the images data into NumPy 
array for both training and testing datasets and converting the image names from pandas series to NumPy array corresponding to each image and assign class number 
based on category of pigmented skin lesion in the dataset. Furthermore, the training and testing datasets were serialised into 
dictionary in a pickle encoded file. Therefore, the encoded file sizes are compact and are portable
in comparison to storing actual image files.
\subsection{Data Normalisation}
The images RGB images numpy array was mutli-demensional array with numbers ranging from 0 to 255.
The numpy array was converted into the float32 format and each element of the array was divided by 255 to normalise the data so, that 
it only ranges between 0 and 1 in float format which will help while training the model. In addition, one hot encoding 
was performed on class labels of the pigmented lesions. The one hot encoding is a representation of categorical variable 
as binary vector and normalise the categorical labels into binary vector.
\pagebreak
\subsection{ Image Segmentation }
\begin{center}
	\includegraphics[width=10cm]{Images/bseg.png}
\end{center}

The figure above reflects the sample from training dataset before performing image segmentation.
The image segmentation was performed on the all the images using binary thresholding in OpenCV framework.

\section{Thresholding Segmentation Algorithm}

Thresholding is one of the commonly adopted method in image segmentation which helps in descrimination most 
significant pixels in the images \citep{al2010image}. The thresold value is selected and the gray scale images  
are converted into the binary representation of the image and value of image which are greater than the thresold
value will be selected with keeping all the attributes of the images such as position and shape \citep{al2010image}. 
Thus, reducing the complexity of the image data and making it easier for classification related tasks. Futhermore, the 
segmented images will be consumed in the model training. The thresholding segmentation was performed using OpenCV library using 
\url{cv.threshold(image, 0.5, 1, cv.THRESH_BINARY)} where threshold value of 0.5 and maximum value of the pixel can be 1
as it was normalised.

\begin{center}
	\includegraphics[width=10cm]{Images/aseg.png}
\end{center}

The figure above shows the result of applying the threshold image segmentation on pigmented skin lesions. 

\pagebreak
\section{Convolutional Model Training}
\begin{figure}[!htp]
    \centering
    \includegraphics[height=.8\textheight]{Documents/model.pdf}
    \caption{Model Architecture 1}
\end{figure}
The model architecture above was implemented using the keras Api in which 2 dimensional 
convolutional layers were added to the sequantial network with intial shape of image in (224, 224, 3) width and height of input image is 224 pixels and the rgb channels depth 
are represented by 3. The initial convolutional layers contains 32 input filters with the kernel size of (3, 3) with relu 
activation function. In the network layers followed by first two convolutional layers are polling layer in the 
architecture MaxPooling was used to extract maximum of the input features after applying image filter or kernel 
to the given image of pigmented skin lesions. Furthermore, dropout of 0.4 was used in the network to generialise the 
overall performance of the network and avoid overfitting of data points. 
The next two layers in the networks are also another convolutional layers with 64 image filters and similar relu activation function. Similar fashion as aboved was 
applied to the network with MaxPooling to extract most significant pixels from feature maps followed by the dropout in the network to generialise it.
The features extracted by the convolutional layers are flattened into one dimensional array. The flattened array will be passed to the fully connected layers in the neural network
to process the information. The model contains three dense hidden layers and one output layer in the neural network.
Furthermore, the model architecture was compiled using various hyper-parameters which effects such as learning rate and 
optimiser for the convolutional model which helps in computing the gradient for the loss function to minimise the error in predecting 
category of pigmented skin lesion.

\subsection{CodePRo}

\pagebreak
\section{Experiments with Hyperparameters}
The experiments were performed on the convolutional model to understand the effects of the hyper-parameters 
such as learning rate of the network, the number of epochs for which the model for trained and increasing the hidden  
and convolutional layers. The adjustments to the hyperparameters which were made on the model had an impact on the accuracy of the overall classification of the pigmented skin lesions.

\subsection{Increased convolutional layers}

\begin{figure}[!htp]
    \centering
    \includegraphics[width=\textwidth]{Images/iConv.png}
    \caption{Convolutional Neural Network}
    \label{fig:cnn}
\end{figure}

The model architecture was changed with increase of four more convolutional layers in the network
to detect more feature maps from images. The additional convolutional layers contains filters of 128 and 256 
respectively.Furthermore, the model was trained for three hundred epochs with validation 
data to adjust the weights and improve the accuracy of the model with learning rate of 0.001 using SGD optimiser from the 
above experiments. The figure \ref{fig:cnn} shows the graph of increase in model accuracy of training and 
validation data over three hundred epochs. Furthermore, the diagram also shows the decline in the 
loss or cost function of the model. The model was evaluated on the testing data and resulted in 77.4 \% accuracy
in detecting the pigmented skin lesions. \footnote{\url{https://github.coventry.ac.uk/sareenv/Final-Year-Project/blob/master/Research}}
\pagebreak
\subsubsection{Model Training without validation data}
\begin{figure}[!htp]
    \centering
    \includegraphics[width=\textwidth]{Images/wvalid.png}
    \caption{Model training without validation data}
    \label{fig:cnnwvalid}
\end{figure}
The experiment was performed with the same model architecture as above without providing the validation 
data to the model. The figure \ref{fig:cnnwvalid} show the increase in the model accuracy and decline in the 
the loss function over three hundred epochs.The accuracy of the model was evaluated to be 74.39\% on testing data. 
Therefore, the further model experiments were performed with validation data. 

\subsection{Learning Rate Experiment}
\begin{figure}[!htp]
    \centering
    \includegraphics[width=15cm]{Images/lr.png}
    \caption{Variable Learning Rates}
    \label{fig:lrates}
\end{figure}

The figure \ref{fig:lrates} above shows that the model was trained for different learning rates as shown
in the legend of the figure \ref{fig:lrates} for thirty epochs or iterations. The outcome of the above test 
was that the model model accuracy of the model were increasing with decrease in the learning rates. 
The graph obtained shows that the model accuracy was proportional to the learning rate while using SGD optimiser. 
The figure \ref{fig:lrates} also shows the decline in the loss function for different learning rates in 
convolutional model trained with SGD optimiser. Therefore, the most optimal learning rate to train the convolutional network 
neural network was 0.1 and 0.001. During the experiments, it was observed that model with the lower the learning rate consumes more time in the 
process of training. Further model training experiments are peformed on 
learning rate of 0.01 to achieve the accuracy and realtive speed to train the model.

\subsection*{Learning Rates Accuracy Results}

\begin{center}
    \begin{tabular} { | c | c | c | c | c |}
        \hline
        Training Time & Learning Rate & Test Accuracy & Epochs  & Optimiser\\ 
        \hline
        02:31:15 & 0.1 & 73.3\% & 140 & SGD \\ 
        \hline 
        02:32:52 & 0.01 & 81.01\% & 140 & SGD  \\
        \hline 
        02:33:00 & 0.001 & 69.75\% & 140 & SGD \\
        \hline
        02:33:22 & 0.0001 & 72.7\% & 140 & SGD \\
        \hline
        02:33:39 & 0.00001 & 72.7\% & 140 & SGD \\
        \hline
    \end{tabular}
\end{center}

The results presented above were evaluated on testing data and shows the 
direct relation of the time required to train convolutional neural network 
where decreasing learning rates requires more time to train models. These results 
are obtained by traing the model on Nvidia GTX-1080 gpu(graphical processing unit), training 
time might differ on different gpu based on it computational power.

\subsection{Epochs Experiment}
\begin{figure}[!htp]
    \includegraphics[width=\textwidth]{Images/epochs.png}
    \caption{Results obtained different epochs}
    \label{fig:epochsTest}
\end{figure}

The figure \ref{fig:epochsTest} shows the accuracy of model with same architecture for 
different number of epochs. The model accuracy is directly proptional the number of epochs as 
training the nural networks is optimisation problem and objective is to find minimum of the cost 
function. The model accuracy on training data improves over each epoch as the model finds 
local minimum of cost function at each epoch and improve the predection in the next iteration.
However, when the model reaches the gloabal minimum of the objective function there will 
be improvments in model accuracy. The figure \ref{fig:epochsTest} shows the model trained 
with the most number of epochs which is eighty in these experiments has least value of 
cost function and maximum training accuracy.

\begin{center}
    \begin{tabular} { | c | c | c | c |}
        \hline
        Learning Rate & Test Accuracy & Epochs  & Optimiser\\ 
        \hline
        0.01 & 73.61\% & 20 & SGD \\ 
        \hline 
        0.01 & 76.97\% & 40 & SGD  \\
        \hline 
        0.01 & 79.46\% & 60 & SGD \\
        \hline
        0.01 & 78.68\% & 80 & SGD \\
        \hline
    \end{tabular}
\end{center}

The table above shows the results obtained from evaluating model accuracy on the test data trained 
over different number of epochs. The general trend can be observed that with the increase in the epochs 
model accuracy was also observed to be improved. 

\section{Segmented Model Training}
\begin{figure}[!htp]
    \centering
    \includegraphics[width=\textwidth]{Images/segmented.png}
    \caption{Segmented Model Training}
    \label{fig:segmodel}
\end{figure}
The model training was performed using the segmented images of pigmented skin lesions as mentioned in the 
image processing section. The model with same architecture was trained under thirty epochs using SGD optimiser. The experiment performed above resulted in 74.0\% model 
accuracy on testing data. The figure \ref{fig:segmodel} shows the increase in accuracy rate of training and 
validation data and decline in the loss function.

\pagebreak
\section{Transfer Learning}
Transfer learning is a method to consume network weights from previously trained model and applying the model 
weights in training of new convolutional neural network. The objective of the experiment was to consume the 
weights obtained from training of model with segmented data and use it in the processing of training same model 
architecture using normal pigmented skin lesions. 
\begin{figure}[!htp]
    \centering
    \includegraphics[width=\textwidth]{Images/transferLearning.png}
    \caption{Transfer Learning Model}
    \label{fig:translearning}
\end{figure}

The figure \ref{fig:translearning} shows the graph indicating increasing in the model accuracy for training 
and validation data and decline in the loss functoin of the model. The model was trained using the SGD optimiser 
for 300 epochs and had produced the accuracy of 79.02\% on the evaluation data. The result of applying the 
transfer learning can be observed as without applying the weights from segmented model training the accuracy 
was evaluated to be 77.4\%. Further experiments are performed using the transfer learning to evaluate the 
impact on model performance.
\pagebreak
\section{VGG16 Architecture}
\begin{figure}[!htp]
    \centering
    \includegraphics[width=\textwidth]{Images/vgg16.png}
    \caption{VGG16 Model Architecture}
\end{figure}

The VGG16 model is heavy model architecture containing 16 layers in the 
model. The model architecture consist of 13 convolutional, 5 pooling and 3 fully connected 
layers. Due to the heavy model architecture with estimated 138 million parameters
and limited hardware resources it was not possible to trian the model on personal machine. As, a result the google colab platform was used develop
the convolutional network \footnote{\url{https://colab.research.google.com/drive/1QHIHhu28sSiW0T-ZGTwg0bp-bulW30l1}}.

\begin{figure}[!htp]
    \centering
    \includegraphics[width=\textwidth]{Images/vgg16Results.png}
    \caption{VGG16 Model Architecture}
    \label{fig:vggRes}
\end{figure}
The figure \ref{fig:vggRes} shows the graph of model accuracy increasing over the 
epochs and decline in the loss function. The VGG model was trained using the SGD optimiser and learning 
rate of 0.01 which resulted in accuracy of 78.14\% over the 60 epochs. 
\subsection{Transfer Learning with VGG16}
Another, test was performed to evaluate the performance of the model with transfering the
weights obtained from the model mentioned above for 300 epochs.

\begin{figure}[!htp]
    \centering
    \includegraphics[width=\textwidth]{Images/vgg162.png}
    \caption{VGG16 Model Architecture}
    \label{fig:vggRes2}
\end{figure}

In the figure \ref{fig:vggRes2} the training and validation accuracy 
was increasing with same growth with increase in the number of epochs.
The result obtained from the experiment was 77.48\% on testing data which is 
worst then the evaluation of the previous model. The poor model accuracy could be a result 
of overfitting of the convolutional network.

\pagebreak
\section{Lenet Architecture}

\pagebreak
\section{Model Deployment}
