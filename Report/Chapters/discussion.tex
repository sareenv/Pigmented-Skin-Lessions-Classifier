\section{Achievement}
The study has succeeded in developing a deep learning-based solution to detect skin cancers from pigmented skin lesions.
Different model experiments were performed to analyse the impact of architecture and hyperparameters on the accuracy of the model to identify the problem. 
The dataset was divided into separate testing data samples which were used for only evaluation purpose. The current research purpose model with the highest accuracy of 82\% performed on 11,00 data samples of pigmented skin lesions. 
The comparison was performed on the diagnosis by an automated system and medical professional to understand the time efficiency and reliability of the automated system. 
The results obtained from performing the comparison has shown that the automated system is significant time-efficient to perform diagnosis. 
In addition, the model has performed diagnosis to detect skin cancer from dermatoscopic images of pigmented skin lesion better than medical practitioners in some cases.
Furthermore, the intelligent model was also deployed on the web-based system to be used by the general audience. The outcome of the research has helped me understand the functioning of convolutional neural networks in visual recognition of pigmented skin lesions.