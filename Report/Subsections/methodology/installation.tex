The research purposes a solution based on deep convolutional neural networks
With various experiments mentioned in further sections.  The experiments
require specific configuration for cuda libraries to run programs on available GPU(Graphical Processing Unit) for faster processing
and further, instructions to setup environment can be found at Tensorflow gpu installation guide \footnote[1]{\url{https://www.tensorflow.org/install/gpu}}.
Alternatively, the machine learning models can also run without installing cuda libraries on CPU but the processing will be slow.
In addition, the convolutional neural network was implemented using Python3 and Jupyter notebooks were used in these experiments which provides an appropriate interface to experiment and write markdowns.  
Jupyter notebooks can be installed by installing Anaconda \footnote[2]{https://www.anaconda.com/distribution/}.
Furthermore, data science and machine learning libraries are required such as Numpy, Keras, Matplotlib and OpenCV. 
Numpy library is required for performing mathematical operations on multi-dimensional NumPy arrays. The matplotlib library provides an interface to visualise the output results. 
The OpenCV library was used for image processing in the process of developing an automated system for classifying
pigmented skin lesions. Moreover, keras library was used to develop deep convolutional models. 
HAM,1000 (Human Against Machine, 1000) dataset was used to develop image classifier \citep{DVN/DBW86T_2018}.
The dataset primarily contains two folders which includes overall 10,000 dermatoscopic images of pigmented skin lesion. In addition, it also contains a csv file which includes the meta-information regarding pigmented skin lesions.
The required files can be downloaded from harvard dataverse webiste \footnote[3]{\url{https://dataverse.harvard.edu/dataset.xhtml?persistentId=doi:10.7910/DVN/DBW86T}}.
