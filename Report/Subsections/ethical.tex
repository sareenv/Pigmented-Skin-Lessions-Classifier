The research project was conducted with ethical considerations where safety and well being of researcher and participants was considered seriously. The Ethics Board has approved the research project for developing a convolutional neural network to detect pigmented skin lesions with ethics id P101878. Due to the current pandemic spread of coronavirus, it was not safe to visit hospitals. As a result, the data collection was performed through the online exchange of forms. The research project involves detecting skin cancer from pigmented skin lesions for which publicly available dataset was used with any personal identity of any patient information which might have ethical considerations to collect data from medical professionals. 
The participants of the study were completely aware of the purpose of the study. The data collection process of secondary research was compliant with GDPR privacy laws in the united kingdom. The documents containing information collected from participants were stored in the password-protected the document and will be destroyed after the research submission. The participants of the research project were informed that they could withdraw their data at any time without any reason or mental pressure. The consent formed was attached to the questionnaire, which required participants to give permissions to collect the data and related concerns.

\section{Research Challenges}
There were various challenges while conducting the research to develop and compare 
the automated system for classification of pigmented skin lesions. The risk of no 
prior knowledge about the convolutional neural network was present in the initial phase 
which required the personal research on foundations of artifical neural network and 
learning to operate the keras deep learning library. The research involved performing 
the image processing and normalisation tasks which requires the superior hardware 
the 8 gigabytes of the RAM (random access memory) was not suffient to normalise the 
original images from the dataset and train the VGG16 network. As, a result the images were resized to 224 * 224 pixels to 
reduce the consumption of the primary memory. Furthermore, the VGG16 model was 
trained on the google colab platform which provides free computation power to 
train the artifical neural networks. However, the colab platform requires the user  to 
interact with the web interface to avoid timeout error. The secondary research for the 
project involves comparing the data collected from the medical professionals with 
automated system. However, due to the pandemic coronavirus it was not safe to 
collect the data from medical hosiptals which resulted in fewer resulted in fewer 
responses from the medical professionals.
