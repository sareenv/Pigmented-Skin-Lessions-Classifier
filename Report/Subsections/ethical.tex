The research project was conducted with ethical considerations where safety and well being of researcher and participants was considered seriously. The Ethics Board has approved the research project for developing a convolutional neural network to detect pigmented skin lesions with ethics id P101878. Due to the current pandemic spread of coronavirus, it was not safe to visit hospitals. As a result, the data collection was performed through the online exchange of forms. The research project involves detecting skin cancer from pigmented skin lesions for which publicly available dataset was used with any personal identity of any patient information which might have ethical considerations to collect data from medical professionals. 
The participants of the study were completely aware of the purpose of the study. The data collection process of secondary research was compliant with GDPR privacy laws in the united kingdom. The documents containing information collected from participants were stored in the password-protected the document and will be destroyed after the research submission. The participants of the research project were informed that they could withdraw their data at any time without any reason or mental pressure. The consent formed was attached to the questionnaire, which required participants to give permissions to collect the data and related concerns.
