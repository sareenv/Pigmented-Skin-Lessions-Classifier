\begin{figure}[!htb]
	\centering
	\includegraphics[width=10cm]{Images/bseg.png}
	\caption{Pigmented Skin lesions before Image Segmentation}
	\label{fig:before_seg}
\end{figure}

Figure \ref{fig:before_seg} shows the sample from training dataset before performing image segmentation.
The figure \ref{fig:before_seg} above was plotted using matplotlib library.


\subsection{Thresholding Segmentation Algorithm}
Thresholding is one of the commonly adopted method in image segmentation which helps in descrimination most 
significant pixels in the images \citep{al2010image}. The thresold value is selected and the gray scale images  
are converted into the binary representation of the image and value of image which are greater than the thresold
value will be selected with keeping all the attributes of the images such as position and shape \citep{al2010image}. 
Thus, reducing the complexity of the image data and making it easier for classification related tasks. Futhermore, the 
segmented images will be consumed in the model training. The thresholding segmentation was performed using OpenCV library using 
\url{cv.threshold(image, 0.5, 1, cv.THRESH_BINARY)} where threshold value of 0.5 and maximum value of the pixel can be 1
as it was normalised.

\begin{figure}[!htp]
	\centering
	\includegraphics[width=10cm]{Images/aseg.png}
	\caption{Binary Images Segmentated Pigmented Skin lesions}
	\label{fig:after_seg}
\end{figure}

The figure \ref{fig:after_seg} shows the result of applying the threshold image segmentation on 
pigmented skin lesions. The process of image segmentation will simplfy the process of 
border and edge detection by the convolutional neural network. Figure \ref{fig:after_seg} was 
plotted using matplotlib library.

\footnote{\url{https://github.coventry.ac.uk/sareenv/Final-Year-Project/blob/master/Research}}
