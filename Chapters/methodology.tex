The research purposes a solution based on deep convolutional neural networks with various experiments mentioned in further sections.
The operations require machine learning specific configuration for Cuda libraries and configuration to run programs on available GPU(Graphical Processing Unit) for faster processing and further, 
instructions to setup environment can be found at \url{https://www.tensorflow.org/install/gpu}. In addition, convolutional neural network was implmented using Python3 and jupyter notebooks were used in these experiments 
which provides an appropriate interface to experiment and write markdowns.
The data science and machine learning libraries such as NumPy for numerical operations, matplotlib for visualisation, Keras for developing deep learning models and OpenCV for image processing was used in the process of developing an automated system for classifying pigmented skin lesions.
HAM,1000 dataset was used to train image classifier, which can be found at
\url{https://dataverse.harvard.edu/dataset.xhtml?persistentId=doi:10.7910/DVN/DBW86T}. The dataset contains two folders containing dermatoscopic images of pigmented skin lesion and CSV file which contains meta information and image name for each pigmented skin lesion.

\section{Data Processing and Normalisation}
The information was read using pandas into the data frame, 
which is a data structure that allows storing tabular data from CSV files. 
Datasets contain unclear and hairy images of pigmented skin lesions which were manually 
removed from the dataset to enhance the quality of available data.
The CSV file contained irrelevant information such as gender and age of 
patients and unbalanced data classes data columns were dropped from the dataset. Furthermore, the dataset was divided into training and 
testing sets using \url{sklearn.model_selection.train_test_split} class in the portion of 80 per cent for 
the training dataset and 20 per cent of testing datasets. The next step towards to preparing the dataset was reading the images data into NumPy 
array for both training and testing datasets and converting the image names from pandas series to NumPy array corresponding to each image and assign class number 
based on category of pigmented skin lesion in the dataset. Furthermore, the training and testing datasets were serialised into 
dictionary in a pickle encoded file. Therefore, the encoded file sizes are compact and are portable
in comparison to storing actual image files.
\subsection{Data Normalisation}
The images RGB images numpy array was mutli-demensional array with numbers ranging from 0 to 255.
The numpy array was converted into the float32 format and each element of the array was divided by 255 to normalise the data so, that 
it only ranges between 0 and 1 in float format which will help while training the model. In addition, one hot encoding 
was performed on class labels of the pigmented lesions. The one hot encoding is a representation of categorical variable 
as binary vector and normalise the categorical labels into binary vector.
\subsection{ Image Segmentation }
\begin{center}
	\includegraphics[width=10cm]{Images/bseg.png}
\end{center}

The figure above reflects the sample from training dataset before performing image segmentation.
The image segmentation was performed on the all the images using binary thresholding in OpenCV framework.

\section{Thresholding Segmentation Algorithm}

Thresholding is one of the commonly adopted method in image segmentation which helps in descrimination most 
significant pixels in the images \citep{al2010image}. The thresold value is selected and the gray scale images  
are converted into the binary representation of the image and value of image which are greater than the thresold
value will be selected with keeping all the attributes of the images such as position and shape \citep{al2010image}. 
Thus, reducing the complexity of the image data and making it easier for classification related tasks. Futhermore, the 
segmented images will be consumed in the model training. The thresholding segmentation was performed using OpenCV library using 
\url{cv.threshold(image, 0.5, 1, cv.THRESH_BINARY)} where threshold value of 0.5 and maximum value of the pixel can be 1
as it was normalised.

\begin{center}
	\includegraphics[width=10cm]{Images/aseg.png}
\end{center}

The figure above shows the result of applying the threshold image segmentation on pigmented skin lesions. 


