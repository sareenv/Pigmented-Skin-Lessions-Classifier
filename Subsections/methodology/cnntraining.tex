\begin{figure}[!htp]
    \centering
    \includegraphics[height=.8\textheight]{Documents/model.pdf}
    \caption{Model Architecture 1}
\end{figure}
The model architecture above was implemented using the keras Api in which 2 dimensional 
convolutional layers were added to the sequantial network with intial shape of image in (224, 224, 3) width and height of input image is 224 pixels and the rgb channels depth 
are represented by 3. The initial convolutional layers contains 32 input filters with the kernel size of (3, 3) with relu 
activation function. In the network layers followed by first two convolutional layers are polling layer in the 
architecture MaxPooling was used to extract maximum of the input features after applying image filter or kernel 
to the given image of pigmented skin lesions. Furthermore, dropout of 0.4 was used in the network to generialise the 
overall performance of the network and avoid overfitting of data points. 
The next two layers in the networks are also another convolutional layers with 64 image filters and similar relu activation function. Similar fashion as aboved was 
applied to the network with MaxPooling to extract most significant pixels from feature maps followed by the dropout in the network to generialise it.
The features extracted by the convolutional layers are flattened into one dimensional array. The flattened array will be passed to the fully connected layers in the neural network
to process the information. The model contains three dense hidden layers and one output layer in the neural network.
Furthermore, the model architecture was compiled using various hyper-parameters which effects such as learning rate and 
optimiser for the convolutional model which helps in computing the gradient for the loss function to minimise the error in predecting 
category of pigmented skin lesion.

\subsection{CodePRo}